\documentclass[journal]{IEEEtran}
\usepackage[utf8]{inputenc} %para más paquetes
\usepackage{cite}
\usepackage{amsmath,amsfonts}
\usepackage{algorithmic}
\usepackage{array}
\usepackage[caption=false,font=normalsize,labelfont=sf,textfont=sf]{subfig}
\usepackage{textcomp}
\usepackage{stfloats}
\usepackage{url}
\usepackage{verbatim}
\usepackage{graphicx}
\graphicspath{{../figures/}}
\hyphenation{op-tical net-works semi-conduc-tor IEEE-Xplore}
\def\BibTeX{{\rm B\kern-.05em{\sc i\kern-.025em b}\kern-.08em
    T\kern-.1667em\lower.7ex\hbox{E}\kern-.125emX}}
\usepackage{balance}
\usepackage[spanish]{babel}

%\renewcommand{\thesection}{}
% para que babel no afecte las subsecciones
\renewcommand{\thesubsection}{\Alph{subsection}}
%\renewcommand{\thesection}{\Alph{section}}




% COMIENZO DEL DOCUMENTO
\begin{document}

\title{Análisis Comparativo del uso de Evolv Rehab para la mejora de miembros inferiores en pacientes con ACV : 4 casos}
\author{Alvaro Raúl Quispe Condori}
%\thanks{Hago mis agradecimientos}}

%ocultar el encabezado, aunque lo vea bonito
%\markboth{Journal of \LaTeX\ Class Files,~Vol.~18, No.~9, September~2020}%
%{How to Use the IEEEtran \LaTeX \ Templates}

\maketitle

\begin{abstract}
   
   
   
   
\end{abstract}

\begin{IEEEkeywords}
Class, IEEEtran, \LaTeX, paper, style, template, typesetting.
\end{IEEEkeywords}


\section{Introducción}
\section{KEYWORD 1 (MARCO TEÓRICO)}
\subsection{Sustema 1}
\subsection{Sustema 2}
\subsection{Sustema 3}

\section{KEYWORD 2 (MARCO TEÓRICO)}
\subsection{Sustema 1}
\subsection{Sustema 2}
\subsection{Sustema 3}


\section{Metodología}
\subsection{Caso 1}
Institución y lugar

Participantes: cantidad de participantes (Grupo de control y grupo experimental), edades, enfermedad, genero.

Tecnología utilizada: software/hardware 

Protocolo o intervención: cuantas sesiones, duración de sesiones, cuantas semanas o meses?

Instrumentos: como se evaluó el resultado: test clínicos o test de ingeniería 

\subsection{Caso 2}
Institución y lugar

Participantes: cantidad de participantes (Grupo de control y grupo experimental), edades, enfermedad, genero.

Tecnología utilizada: software/hardware 

Protocolo o intervención: cuantas sesiones, duración de sesiones, cuantas semanas o meses?

Instrumentos: como se evaluó el resultado: test clínicos o test de ingeniería 


\subsection{Caso 3}
Institución y lugar \cite{oxford}

Participantes: cantidad de participantes (Grupo de control y grupo experimental), edades, enfermedad, genero.

Tecnología utilizada: software/hardware 

Protocolo o intervención: cuantas sesiones, duración de sesiones, cuantas semanas o meses?

Instrumentos: como se evaluó el resultado: test clínicos o test de ingeniería 


\subsection{Caso 4}
Contexto (institución y lugar), tiempo que duró la experiencia, cantidad de participantes, resultados obtenidos, software/hardware utilizado, metodología utilizados (cuantas sesiones, duración de sesiones, cuantas semanas o meses) 

\section{Resultados y Discución}
\subsection{Diferencias}
\subsection{Discución}
\subsection{Propuesta}
\subsection{Semejanzas}






% lo que hace es tratar de igualar el final de cada página 
\balance

\bibliographystyle{IEEEtran}
\bibliography{refs}



% para briografía
%\begin{IEEEbiographynophoto}{Jane Doe}
%Biography text here without a photo.
%\end{IEEEbiographynophoto}


\end{document}


